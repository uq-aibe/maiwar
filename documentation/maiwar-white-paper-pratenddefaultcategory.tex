

\makeatletter\Hy@SaveLastskip\label{proofsection:prAtEndii}\ifdefined\pratend@current@sectionlike@label\immediate\write\@auxout{\string\gdef\string\pratend@section@for@proofii{\pratend@current@sectionlike@label}}\fi\Hy@RestoreLastskip\makeatother\begin{proof}[Proof of \pratendRef{thm:prAtEndii}]\phantomsection\label{proof:prAtEndii}\label {proof-obs-c2d} This follows directly from \cref {prop-c2dQ} and the construction of $\preceq _{\mbbj }$.\end{proof}

\makeatletter\Hy@SaveLastskip\label{proofsection:prAtEndiii}\ifdefined\pratend@current@sectionlike@label\immediate\write\@auxout{\string\gdef\string\pratend@section@for@proofiii{\pratend@current@sectionlike@label}}\fi\Hy@RestoreLastskip\makeatother\begin{proof}[Proof of \pratendRef{thm:prAtEndiii}]\phantomsection\label{proof:prAtEndiii}\label {proof-obs-c2d} Respectively, these two statements follow via \cref {prop-central-testworthy} and \cref {lem-test-empty-fourdiv}.\end{proof}

\makeatletter\Hy@SaveLastskip\label{proofsection:prAtEndiv}\ifdefined\pratend@current@sectionlike@label\immediate\write\@auxout{\string\gdef\string\pratend@section@for@proofiv{\pratend@current@sectionlike@label}}\fi\Hy@RestoreLastskip\makeatother\begin{proof}[Proof of \pratendRef{thm:prAtEndiv}]\phantomsection\label{proof:prAtEndiv}This follows from \cref {thm-foureq}, \cref {lem-induction} and the fact that, via \cref {lem-test-empty-fourdiv}, \fourpru \ implies \ref {T}--\ref {A} when the number of case types is finite. \label {proof-cor-foureq}\end{proof}

\makeatletter\Hy@SaveLastskip\label{proofsection:prAtEndv}\ifdefined\pratend@current@sectionlike@label\immediate\write\@auxout{\string\gdef\string\pratend@section@for@proofv{\pratend@current@sectionlike@label}}\fi\Hy@RestoreLastskip\makeatother\begin{proof}[Proof of \pratendRef{thm:prAtEndv}]\phantomsection\label{proof:prAtEndv}Via \cref {lem-axiomsQ}, \ref {K}--\ref {A} and \fourdiv \ hold for $\preceq _{\mbbd }$ if and only if \ref {KQ}--\ref {AQ} and \fourdiv \ hold for $\preceq _{\mbbj }$. Let $Y\subseteq X$ be of cardinality $m' = 1, 2, 3$ or $4$ and let $\ext $ be a regular $Y$-extension. Via \cref {lem-insep}, there exists a pairwise representation $v^{\dd } $. For $m' = 1$, $n' = 1 $ because $\ext _{J}$ is constant on $\mbbjp $. For $m' = 2$, $n' \geq 2$, since via part \ref {D-insep} of \cref {lem-insep}, $G^{\xy }$ and $G^{\yx }$ are both nonempty. \par By way of contradiction, first suppose $n' = 2$ and $m' \geq 3$. Via \cref {eg-zaslavski} of appendix \ref {sec-proof-mainQ}, $\total (\ext ) \leq 4$. In contrast, \fourdiv \ requires $\total (\ext ) = 6$. The remaining case is where $n' = 3$ and $m' \geq 4$. If the rank $\mathbf r$ of $v^{\dd }$ satisfies $\mathbf r \geq 3$, then the kernel $A^{Y}$ of $v^{\dd }$ is zero-dimensional. Then $0$ is the unique element of $ A^{Y}$. Thus, the positive kernel $A^{Y}_{\mdoubleplus }$ of $v^{\dd }$ is empty. Then Zaslavski's theorem implies that $\total (\ext )< 4!$, so that \fourdiv \ fails to hold. If $\mathbf r \leq 2$, then an application of the rank version Zaslavski's theorem (in particular \cref {eq-zaslavski-4} with $\acute {\mathbf r } = \mathbf r = 2$) yields \begin {linenomath*} \[ \lvert \mc G_{\mdoubleplus } \rvert \leq 1 - 6 + 15 - 20 + 15 + 6 + 1 = 12. \] \end {linenomath*} Thus, once again \fourdiv \ fails to hold. Thus $n' \geq \min \{4, m'\}$, as required. Finally, since $Y\subseteq X$, $m\geq m'$, and, via part \ref {item-dimension} of \cref {def-extension}, $n \geq n'$.\end{proof}

\makeatletter\Hy@SaveLastskip\label{proofsection:prAtEndvi}\ifdefined\pratend@current@sectionlike@label\immediate\write\@auxout{\string\gdef\string\pratend@section@for@proofvi{\pratend@current@sectionlike@label}}\fi\Hy@RestoreLastskip\makeatother\begin{proof}[Proof of \pratendRef{thm:prAtEndvi}]\phantomsection\label{proof:prAtEndvi}\label {proof-nov-iso} \par We show that there exists a canonical embedding (a structure preserving injection) of $ \nov ( Y , \preceqb _ \mbbj ) $ into $ \nov ( Y , \preceqb _ \mbbd ) $. The fact that this map is also surjective follows from the fact that $ \nov ( Y , \preceqb _ \mbbd ) $ can be embedded in $ \nov ( Y , \preceqb _ \mbbj ) $ in precisely the same way. The proof that the two sets of regular extensions are isomorphic follows via a similar argument plus the observation that every $Y$-extension is either regular or novel. \par Take $ \extb \in \nov ( Y , \preceqb _ \mbbj )$ and define $ \hextb = \langle \hextb _{C} : C \in \mbbdp \rangle $ via the property: for each $ C \in \mbbdp $, $ \hextb _{C}\defeq \extb _{J} $ if, and only if, $ L_{C}= L_{J} $, where, as before, $t \mapsto L_{C}(t) $ counts the number of cases of type $t$ in $C$ and $ L_{J} = \kappa _{J} J \in \mbbip $ for some minimal $\kappa _{J} \in \nnint $. Now, for any $ \extb ' \neq \extb $ in $ \nov ( Y , \preceqb _ \mbbj ) $, there exists $J\in \mbbjp $ such that $\extb '_{J}\neq \extb _{J}$. If we define $ \hext ' $ analogously, so that it is equivalent to $\ext '$, then $ \hextb ' \neq \hextb $. As a consequence, the canonical mapping $\extb \mapsto \hextb $ is injective. If we can show that $ \hext $ does in fact belong to $ \nov ( Y , \preceqb _ \mbbd )$, then we have constructed the required embedding. The fact that $ \hext $ satisfies \ref {item-binary-rel} and \ref {item-preserving} of \cref {def-extension} follows immediately from \cref {def-extensionQ}. The proof that \cref {item-dimension} of \cref {def-extension} holds is as follows. Take any $ c , c ' \in \mbbcp $ and $ D \in \mbbdp $ such that $ c \sim ^ \star c ' $ and $ c , c ' \notin D $. First, observe that $ D \cup \{c\} \sim ^ \star D \cup \{c'\} $, and moreover, for some $ t \in \mbbtp $ we have $ c , c ' \in t $. Then, for every $t\in \mbbtp $, $ \lvert D \cup \{c\} \rvert = \lvert D \cup \{c'\} \rvert = L$ for some $ L \in \mbbip \cap \mbbjp $. Thus $ \hextb _ { D \cup \{c\} } = \hextb _ { D \cup \{c'\}}$, as required for $ \hext $ to be an extension of $ \preceqb _{ \mbbd }$. Finally, via \cref {def-extensionQ}, the definition of a novel extension ensures that the induced equivalence relation $\sim ^{\extb }$ on $\mbbcp $ satisfies $ c \not \sim ^{\extb } \novel $ for every $ c \in \mbbc $. Since $ \sim ^{ \hextb }$ inherits this property, $ \hext $ is novel.\end{proof}

\makeatletter\Hy@SaveLastskip\label{proofsection:prAtEndvii}\ifdefined\pratend@current@sectionlike@label\immediate\write\@auxout{\string\gdef\string\pratend@section@for@proofvii{\pratend@current@sectionlike@label}}\fi\Hy@RestoreLastskip\makeatother\begin{proof}[Proof of \pratendRef{thm:prAtEndvii}]\phantomsection\label{proof:prAtEndvii}\label {proof-axiomsQ} \par Fix $ \extb _ { \mbbjpp } \equiv \hextb _ { \mbbdpp }$ and assume that $ \hextb _{ \mbbdpp } $ satisfies \ref {C}. We show that $ \extb _ { \mbbjpp }$ satisfies \ref {CQ}. Fix $ x , y \in Y $ and $ J \in \mbbjpp $ such that $ x \ext _{ J } y $ and $ x \ext _{ J '} y $. Fix $ \lambda , \mu \in \posrat $ and let $ \kappa $ be the smallest positive integer such that both $ L \defeq \kappa \lambda J $ and $ L ' \defeq \kappa \mu J ' $ belong to $ \mbbipp $. Then, by \cref {lem-coneQ}, we have both $ x \ext _{ L } y $ and $ x \ext _{ L ' } y $. Moreover, for $D, D'$ such that $L_{D}= L$ and $L_{D'}= L'$ , we have $ x \hext _{ D } y $ and $ x \hext _{ D ' } y $ and, by \ref {C}, $ x \hext _{ D \cup D ' } y $. Finally, since $ L_{D} + L_{D'} = \kappa (\lambda J + \mu J ' )$, one further application of \cref {lem-coneQ} yields $ x \sext _{ \lambda J + \mu J ' } y $, as required for \ref {CQ}. \par \par The proof that “\ref {C} implies \ref {CQ}” is \emph {mutatis mutandis} a special case of the above argument and ommitted. We now assume $ \hextb _{ \mbbdpp } $ satisfies \ref {C} and \ref {A} and prove that $ \extb _{ \mbbjpp }$ satisfies \ref {AQ}. Fix $ x , y \in X $ such that $x \sext _{ J } y $ for some $ J \in \mbbj $ and take any $ J ' \in \mbbjpp $. Then, by the construction of $ \extb _ \mbbjpp $, there exists $ L, L ' \in \mbbi $ such that $ j J = L $ and $ j ' J ' = L ' $ for some $ j , j ' \in \posint $. By \cref {lem-coneQ}, $ \extb _{L} = \extb _{J} $ and $ \extb _ { L '} = \extb _ {J '} $. Moreover, by construction, for some $D$ and $D'$ such that $L_{D}=L$ and $I_{D'}= L'$, $ \hextb _{D} = \extb _ J $ and $ \hextb _ { D'} = \extb _ {J '} $. We therefore conclude that $ x \hsext _{ D} y $, so that \ref {A} implies the existence of $ \kappa \in \posint $ and $\{D_{l}: D_{l}\sim ^{\hextb } D\}_{1}^{\kappa }$ such that $ x \hsext _ { D_{1}\cup \cdots \cup D_{\kappa } \cup D '} y $. Then, by the construction of $ \extb _{ \mbbjpp }$, $ x \sext _ { \kappa L_{D} + L_{D'}} y $. Let $ \nu \defeq \frac { 1 } { \kappa j + j '}$ and take $ \lambda = \nu j '$, so that $0 < \lambda < 0 $ and $ 1-\lambda = \nu \kappa j $. In fact, since $ \lambda \in \mbb Q$, we have \begin {linenomath*} \[ K \defeq (1-\lambda ) J +\lambda J ' \in \mbbjpp .\] \end {linenomath*} Simplifying, we obtain $ K = \nu ( \kappa L + L ')$. Since $ \nu \in \posrat $ and $ \kappa L + L ' \in \mbbjpp $, \cref {lem-coneQ} implies $ \extb _ { K } = \extb _ { \kappa L + L ' }$. This allows us to conclude that $ x \sext _ K y $. Finally, take any $ \mu \in \mbb Q \cap ( 0 ,\lambda )$. From basic properties of the real numbers, there exists $ \xi < 1 $ such that $ \mu = \xi \lambda $ and, moreover, $ \xi $ is rational. Next, note that the definition of $ K $ implies $ \xi ( K - J ) = \xi \lambda ( J ' - J ) $. Adding $ J $ to each side of the latter and applying the definition of $ \mu $ yields \begin {linenomath*} \[(1 - \xi ) J + \xi K = ( 1 - \mu ) J + \mu J ' . \] \end {linenomath*} Then, since $ x \sext _ J y $ and $ x \sext _ K y $, \ref {CQ} implies $x \sext _ { ( 1 - \mu ) J + \mu J '} y $, as required for \ref {AQ}. \par Conversely, we now assume that $ \ext _{ \mbbjpp } $ satisfies \ref {CQ} and \ref {AQ} and prove that \ref {A} holds. Take $ D , D ' \in \mbbd $ such that $ x \hsext _{D} y $ and any other $ D' \in \mbbd $. Let $L=L_{D}$ and $L' = L_{D'}$. Then, by construction, $ x \sext _{L} y $ and, by \ref {AQ}, there exists $\lambda \in \mbb Q \cap ( 0, 1) $ such that $ x \sext _ { (1 -\mu ) L +\mu L'} y $. Then, since $\mu $ is rational, $\mu = \nicefrac { j } { k } $ for some $ j , k \in \posint $. Let $ q : = (1 - \mu ) /\mu = ( k - j )/ j $ and let $ \kappa = j q $, so that $ \kappa = k - j $. The fact that $ 0 < \mu < 1 $ ensures that $ \kappa \in \posint $. To complete the proof, we show that $ x \sext _ { \kappa L + L'} y $, for then the existence of $D_{1}, \dots , D_{\kappa }$ such that $ x \hsext _ { D_{1}\cup \cdots \cup D_{\kappa } \cup D'} $ immediately follows. Together $ x \sext _ { (1 -\mu ) L +\mu L' } y $ and \cref {lem-coneQ} imply $ x \sext _ { q L + L ' } y $. Similarly, together $ x \sext _{ L } y $ and \cref {lem-coneQ} imply $ x \sext _ { ( j - 1 )q L } y $. Then, since $ ( j - 1 )q L + ( q L + L ' ) = j q L + L ' $ and $ \kappa = j q $, an application of \ref {CQ} yields $x \sext _{\kappa L + L '} y $, as required.\end{proof}

\makeatletter\Hy@SaveLastskip\label{proofsection:prAtEndviii}\ifdefined\pratend@current@sectionlike@label\immediate\write\@auxout{\string\gdef\string\pratend@section@for@proofviii{\pratend@current@sectionlike@label}}\fi\Hy@RestoreLastskip\makeatother\begin{proof}[Proof of \pratendRef{thm:prAtEndviii}]\phantomsection\label{proof:prAtEndviii}\par In addition to \ref {KQ}–\ref {AQ}, the proof of lemma 1 of \gsii \ only appeals to \twodiv . That lemma, like the present one, does not require \ref {TQ}, or any diversity condition stronger than \twodiv \ since it is only result about distinct pairs of elements $x$ and $y$ in isolation. \par \Wlog \ we suppress reference to the acute accent. Modulo notation, lemma 1 of \gsii \ and its proof show that \ref {KQ}–\ref {AQ}, in addition to \twodiv \ on $Y$, imply the existence $v^{\dd }$ with rows satisfying properties \ref {K-insep}–\ref {unique-insep} of the present lemma. \par We now prove the converse: that properties \ref {K-insep}–\ref {unique-insep} imply the axioms hold. For \ref {KQ}, fix arbitrary $J \in \mbbj $. For every $x,y \in Y$, the fact that $\langle v^{\xy },\cdot \rangle $ is real-valued and, via property \ref {skew-insep}, equal to $-\langle v^{\yx },\cdot \rangle $, ensures that $J$ belongs to one of the sets $ H_{\mdoubleplus }^{\{x,y\}}$, $G_{\mdoubleplus }^{\xy }$ and $G_{\mdoubleplus }^{\yx }$. Then property \ref {K-insep} and the fact that $J $ belongs to $\mbb Q^{\mbbtpp }$ completes the proof. \ref {CQ} and \ref {AQ} hold by virtue of the fact that $\langle v^{\xy },\cdot \rangle $ is linear on $\posreal ^{\mbbtpp }$. Finally, we prove that \twodiv \ holds. Take any distinct $x,y \in Y$. Then, via property \ref {D-insep}, both $G_{\mdoubleplus }^{\xy } $ and $G_{\mdoubleplus }^{\yx }$ are nonempty. By continuity of $\langle v^{\xy }, \cdot \rangle $, $G_{\mdoubleplus }^{\xy }$ and $G_{\mdoubleplus }^{\yx }$ are also open in $\posreal ^{\mbbtpp }$. As such they each contain a rational vector, so that, by property \ref {K-insep}, \twodiv \ holds on $\{x,y\}$. \par We now prove the characterisation of novel extensions. Let $\ext $ be a novel $Y$-extension with matrix representation $v^{\dd }$ satisfying parts \ref {K-insep}–\ref {unique-insep} of the lemma. Fix arbitrary $ t \neq \novel $. Then \cref {def-extensionQ} implies the existence of $ J \in \mbbj $ and $ L = J \times 0 \in \mbbjp $ such that $ \extb _ { L + \delta _{t} } \neq \extb _ { L + \delta _{\novel } }$. \Wlog , consider the case where, for some $ x , y \in Y $, it holds that both $ y \ext _ { L + \delta _ t } x $ and $ x \sext _ { L + \delta _ \novel } y $. Equivalently, \begin {linenomath*} \[ \langle v^{\xy }, L + \delta _ t \rangle \leq 0 <\langle v^{\xy }, L + \delta _ \novel \rangle \] \end {linenomath*} which, via linearity of $\langle v^{\xy }, \cdot \rangle $, we may rearrange to obtain \begin {linenomath*} \begin {equation}\label {eq-nov} v^{\xy } ( t ) \leq -\langle v^{\xy }, L \rangle < v ^ \xy ( \novel ) . \end {equation} \end {linenomath*} Thus, $v^{\xy } (t) \neq v^{\xy }(\novel )$, as required for the lemma. \par For the converse argument, fix arbitrary $t \in \mbbt $. Then $\acute v^{\dd } (t) \neq \acute v^{\dd } (\novel )$ implies the existence of distinct $x,y \in Y$ such that $ \acute v^{\xy }(t) \neq \acute v^{\xy }(\novel ) $. We show that there exists $J \in \mbbj $ and $L= J\times 0 \in \mbbjp $ satisfying \cref {eq-nov}. For then, by retracing (in reverse order) the arguments that lead to \cref {eq-nov}, we arrive at the conclusion that $ \extb _ { L + \delta _{t} } \neq \extb _ { L + \delta _{\novel }} $, as required for $\ext $ to be novel. \par Take $ \mu = v ^ \xy ( t ) $ and $ \xi = v ^ \xy ( \novel ) $ and consider the case where $ \mu < \xi < 0 $. Since $\mu <0$, \twodiv \ implies that there exists $s\in \mbbt $ such that $v^{\xy }(s)$ is positive. Then, for some $ \lambda \in \posrat $, $ - \lambda v ^ \xy ( s ) \in ( \mu , \xi ) $. Let $ L = \lambda \delta ^{ \novel } _ s $ and observe that \begin {linenomath*} \[ \mu < - \langle v^{\xy }, L \rangle < \xi ,\] \end {linenomath*} as required. \emph {Mutatis mutandis}, the case where both $ \mu $ and $ \xi $ are positive is the same. If $ \mu \leq 0 \leq \xi $, then take $ L = 0 $, so that $\mu \neq \xi $ yields $ \extb _ { L + \delta ^{ \novel } _ t } \neq \extb _ { L + \delta ^{ \novel } _ \novel } $.\end{proof}

\makeatletter\Hy@SaveLastskip\label{proofsection:prAtEndix}\ifdefined\pratend@current@sectionlike@label\immediate\write\@auxout{\string\gdef\string\pratend@section@for@proofix{\pratend@current@sectionlike@label}}\fi\Hy@RestoreLastskip\makeatother\begin{proof}[Proof of \pratendRef{thm:prAtEndix}]\phantomsection\label{proof:prAtEndix}Fix $\countof Y= 3$ or $4$, via \cref {lem-insep}, let $ v^{\dd }$ denote the $2$-diverse matrix representation of the improper $Y$-extension $\ext $. Let $\mc H_{\mdoubleplus }$ denote the associated arrangement of hyperplanes. For every distinct $x,y\in X$, \cref {lem-insep} implies that $H^{\{x,y\}}_{\mdoubleplus }$ intersects $ \posreal ^{\mbbt }$. Then, similar to \cref {eg-zaslavski}, the $ 1 \leq n \leq \binom {\countof Y}{2} $ distinct hyperplanes of $\mc H_{\mdoubleplus } $ cut $\posreal ^{\mbbt }$ into at least $n+1$ regions. At least one pair $ G$ and $G^{*}$ in $\mc G_{\mdoubleplus }$ are therefore separated by all $n$ distinct members of $\mc H_{\mdoubleplus }$. Take $J \in G$, so that, for every distinct $x,y \in Y$, $\langle u^{\xy }, J \rangle \neq 0$. Thus $\ext _{J}$ is antisymmetric, complete and, via \ref {TQ}, total. Next, take $L \in G^{*}$, so that since $J$ and $L$ are separated by every hyperplane in $\mc H_{\mdoubleplus }$, $\extb _{J}= \extb _{L}^{-1}$.\end{proof}

\makeatletter\Hy@SaveLastskip\label{proofsection:prAtEndx}\ifdefined\pratend@current@sectionlike@label\immediate\write\@auxout{\string\gdef\string\pratend@section@for@proofx{\pratend@current@sectionlike@label}}\fi\Hy@RestoreLastskip\makeatother\begin{proof}[Proof of \pratendRef{thm:prAtEndx}]\phantomsection\label{proof:prAtEndx}\label {proof-lem-c2dQ} Let \ref {c2dQ} hold on $Y$. Since $v^{\dd }$ is a $2$-diverse pairwise representation, $v^{\xz }, v^{\yz }\not \leq 0$. By \ref {c2dQ}, one of $G_{\mplus }^{\xz }$ and $G_{\mplus }^{\zx }$ contains both $J,L$ such that \begin {linenomath*} \[\langle v^{(y,z)} , L \rangle <0< \langle v^{(y,z)} , J\rangle .\] \end {linenomath*} \Wlog , suppose $J, L $ belongs to $ G_{\mplus }^{\xz }$. Then $\langle v^{\xz },\cdot \rangle $ is positive on $\{L,J\}$, so that, for every $\lambda \in \R $ $v^{\xz } \neq \lambda v^{\yz }$, as required. \par \par Conversely, let $x,y,z \in Y$ be such that $v^{\xz }$ and $v^{\yz }$ are noncollinear. Then $H_{\mdoubleplus }^{\{x,z\}}\neq H_{\mdoubleplus }^{\{y,z\}}$, and there exists $L \in H_{\mdoubleplus }^{\{x,z\}}\bs H_{\mdoubleplus }^{\{y,z\}}$. \Wlog , therefore, suppose $L \in H^{\{x,z\}}_{\mdoubleplus } \cap G^{\yz }_{\mdoubleplus }$. Since $v^{\dd }$ is $2$-diverse, there exists $s,t\in \mbbt $ such that $v^{\xz }(s)<0<v^{\xz }(t)$. Noting that $L \in \posreal ^{\mbbtpp }$, so that $ v^{\xz } \neq \delta _{s},\delta _{t}$, let $\psi _{s}$ and $\psi _{t}$ be the convex paths from $L $ to $\delta _{s}$ and $\delta _{t}$ respectively. For sufficiently small $\lambda > 0$, $ \langle v^{\yz },\psi _{s'} (\lambda ) \rangle $ remains positive for $s'= s, t$ and, moreover, since $L \in H^{\{x,z\}}_{\mdoubleplus }$, \begin {linenomath*} \[\langle v^{\xz }, \psi _{s}(\lambda )\rangle <0< \langle v^{\xz }, \psi _{t}(\lambda ) \rangle . \] \end {linenomath*} Finally, since $L$ has finite support, a finite sequence of perturbations of the elements of $\psi _{s}(\lambda )$ and $\psi _{t}(\lambda )$ yields (rational-valued) members of $\mbbjpp $ with the same properties, as required for \ref {c2dQ}.\end{proof}

\makeatletter\Hy@SaveLastskip\label{proofsection:prAtEndxi}\ifdefined\pratend@current@sectionlike@label\immediate\write\@auxout{\string\gdef\string\pratend@section@for@proofxi{\pratend@current@sectionlike@label}}\fi\Hy@RestoreLastskip\makeatother\begin{proof}[Proof of \pratendRef{thm:prAtEndxi}]\phantomsection\label{proof:prAtEndxi}\label {proof-prop-c2dQ} When $X=2$, \ref {c2dQ} and \ref {p3dQ} are identical to \twodiv . Let $ Y = \{ x, y , z \} \subseteq X$ and let $\ext $ denote the improper $Y$-extension of $\preceq _{\mbbj }$. We begin by assuming \ref {c2dQ} and showing that $\countof Y + 1 = 4 \leq \countof \total ( \ext )$. Via \cref {lem-c2dQ}, there are three distinct hyperplanes $H_{\mdoubleplus }^{\{x,y\}}, H_{\mdoubleplus }^{\{y,z\}}$ and $H_{\mdoubleplus }^{\{x,z\}}$ in the associated arrangement $\mc H_{\mdoubleplus }$. Then, as in \cref {eg-zaslavski}, $\emptyset = \posreal ^{\mbbt }$ is the unique element of $\mc L_{\mdoubleplus }$ that lies below each member of $\mc H_{\mdoubleplus }$. Thus, via \cref {eq-mobius}, $\bmu (A^{\emptyset }) = 1 $ and $\bmu (A) = - \bmu (A^{\emptyset })$ for all three hyperplanes $A \in \mc H_{\mdoubleplus }$. Thus, Zaslavski's theorem implies that $\countof \mc G_{\mdoubleplus }$ is bounded below by $4$. Thus $\total (\ext ) \geq 4$, and since, for every $Y$-extension $\aext $, $\countof \total (\aext ) \geq \countof \total (\ext )$, \ref {p3dQ} holds. \par Conversely, suppose \ref {p3dQ} holds and, once again let $\ext $ denote the improper $Y$-extension of $\preceq _{\mbbj }$, so that $\total (\ext ) \geq \countof Y = 3$. Now \ref {p3dQ} implies \twodiv , so that, via \cref {lem-insep}, there exists a $2$-diverse matrix representation with associated arrangement $\mc H_{\mdoubleplus }$. It is not the case that $\lvert \mc H_{\mdoubleplus } \rvert = 1$, for this would imply that $\countof \total (\ext ) = 2$. \Wlog , suppose $H_{\mdoubleplus }^{\{x,y\}} \neq H_{\mdoubleplus }^{\{y,z\}}$. Observe that \ref {TQ} then implies $H_{\mdoubleplus }^{\{x,z\}} \neq H_{\mdoubleplus }^{\{x,y\}}$ and $H_{\mdoubleplus }^{\{x,z\}} \neq H_{\mdoubleplus }^{\{y,z\}}$. This implies that $v^{\xy }, v^{\yz } $ and $v^{\xz }$ are pairwise noncollinear. Finally, an application of \cref {lem-c2dQ} then yields \ref {c2dQ}.\end{proof}

\makeatletter\Hy@SaveLastskip\label{proofsection:prAtEndxii}\ifdefined\pratend@current@sectionlike@label\immediate\write\@auxout{\string\gdef\string\pratend@section@for@proofxii{\pratend@current@sectionlike@label}}\fi\Hy@RestoreLastskip\makeatother\begin{proof}[Proof of \pratendRef{thm:prAtEndxii}]\phantomsection\label{proof:prAtEndxii}\label {proof-lem-induction}Note that, when $1 \leq \lvert X \rvert \leq 2 $, \fourjac \ holds vacuously and $\preceq _{\mbbj }$ has a Jacobi representation via \cref {lem-insep}. For general $X$, the fact that \fourjac \ is necessary for $\preceq _{\mbbj } $ to have a Jacobi representation follows simply because if the Jacobi identity holds on $X$, then it holds on every $Y\subseteq X$. For the sufficiency of \fourjac , we proceed by induction. As in lemma 3 and claim 9 of \gsii , we assume that $X$ is well-ordered. \par ^^L \par In the case that $ \lvert X \rvert \leq 4 $, we only need to show that $ v ^{ \dd } $ is unique. \Wlog , we take the initial step in our induction argument to satisfy $ \lvert X \rvert = 4 $. Let $ \mathbf v ^{\dd }$ denote any other Jacobi representation of $\preceq _{\mbbj }$. By \cref {lem-insep}, for every distinct $ x , y \in Y ^{ 2 } $, there exists $ \lambda ^{\{x,y\}}> 0 $ such that $ \mathbf v ^{ \xy } = \lambda ^{\{x,y\}} v ^{ \xy }$. We need to show that $ \lambda ^{\{x,y\}} = \lambda $ for every distinct $ x , y \in Y $. Let $ Y = \{ x , y , z , w \}$. By \cref {lem-c2dQ}, the set $\{ v ^{ \xy } , v ^{ \xz } , v ^{ \xw } \}$ is pairwise noncollinear. Then, since the Jacobi identity holds for both $ v ^{ \dd } $ and $ \mathbf v ^{\dd }$, we derive the equation \begin {linenomath*} \begin {equation}\label {eq-jac-unique} (1 - \lambda ^{\{x,y\} })v ^{ \xy } + (1 - \lambda ^{\{y,z\}})v ^{\yz } = (1 - \lambda ^{\{x,z\}}) v ^{\xz } \end {equation} \end {linenomath*} Suppose that $ 1 - \lambda ^{\{y,z\} } = 0 $. Then, either the other coefficients in \cref {eq-jac-unique} are both equal to zero (and our proof is complete), or we obtain a contradiction of \cref {lem-c2dQ}. Thus, $ 1 - \lambda ^{\{y,z\} } $ is nonzero and we may divide through by this term and solve for $ v ^{ \yz } $. First note that, since $ v ^{ \dd }$ is a Jacobi representation, $ v ^{ \yx }+ v ^{ \xy } = v ^{ (y,y) } = 0$. Then, since $ v ^{ \yx } = - v ^{ \xy }$, \begin {linenomath*} \[ v ^{ \yz } = \frac { 1- \lambda ^{ x y }}{ 1 - \lambda ^{ y z }} v^{ \yx } + \frac { 1 - \lambda ^{ x y }}{ 1 - \lambda ^{ y z }} v^{ \xz }. \] \end {linenomath*} We then conclude that both of the coefficients in the latter equation are equal to one. (This follows from linear independence of $ v ^{ \yx }$ and $ v ^{ \xz }$ together with the Jacobi identity $ v ^{ \yz } = v ^{ \yx } + v ^{ \xz }$.) Thus, $ \lambda ^{\{x,y\}} = \lambda ^{\{y,z\}} = \lambda ^{\{x,z\}} $, as required. Repeated application of the same argument to the remaining Jacobi identities yields the desired conclusion, $ \mathbf v^{\dd }= \lambda v^{\dd }$. \par ^^L \par For the inductive step, take $Y$ to be an initial segment of $X$. By the induction hypothesis, there exists a Jacobi representation $ \v ^{ \dd }: Y^{2} \times \mbbt \rightarrow \R $ of the improper $Y$-extension $\extb = \preceqb _{\mbbj } \cap Y^{2} $ that is suitably unique. \begin {claim}\label {claim-induction-well-defined} For every $ w \in X \bs Y $ and $W \defeq Y \cup \{w\}$, there exists a Jacobi representation $\hat {v}^{\dd }: W^{2} \times \mbbt \rightarrow \R $ of the improper $W$-extension $\hextb $. \end {claim} \begin {proof}[Proof of \cref {claim-induction-well-defined}] Via \cref {lem-c2dQ}, there exists a conditionally $2$-diverse pairwise representation $u^{\dd }$ of $\preceq _{\mbbj }$. Fix any four distinct elements $x, x ' , y , z$ in $Y$. \Cref {lem-insep} implies the existence of $\phi , \phi ' \in \posreal $ such that $\phi u^{\xz } = \v ^{\xz }$ and $\phi ' u^{\xpz } = \v ^{\xpz }$. Let $Z = \{ x, y , z , w\}$ and $Z ' = \{ x' , y , z , w\}$. Since \threejac \^^Mholds, there exist positive scalars $ \alpha , \beta , \hat \beta , \gamma , \sigma $ and $ \tau $ such that \begin {linenomath*} \begin {align}\label {eq-xy} \alpha u ^{ \xw } + \beta u ^{ \wy } &= \gamma u ^{ \xy },\\ \label {eq-yz} \hat \beta u ^{ \yw } + \sigma u^{ \wz } &= \tau u ^{ \yz }, \text { and}\\ \label {eq-xz} \gamma u ^{ \xy } + \tau u ^{ \yz } &= \v ^{ \xz } . \end {align} \end {linenomath*} \addtocounter {linenumber}{-1} Moreover, \fourjac \ ensures that we may take $\beta = \hat \beta $. Since $u^{\dd }$ is conditionally $2$-diverse, $ \{ u ^{ \xy } , u ^{ \yz } \}$ is linearly independent, and the linear system \cref {eq-xz} in the unknowns $ \gamma $ and $ \tau $ has a unique solution. This, together with the induction hypothesis (which yields $ \v ^{ \xy } + \v ^{ \yz } = \v ^{ \xz }$) implies that $ \gamma u ^{ \xy } = \v ^{ \xy }$ and $ \tau u ^{ \yz } = \v ^{ \yz } $. Similarly, for $ Z ' $, \fourjac \ yields $ \alpha ' , \beta ' , \sigma ' , \gamma ' , \tau ' > 0 $ such that \begin {linenomath*} \begin {align}\label {eq-xpy} \alpha ' u ^{ \xpw } + \beta ' u ^{ \wy } &= \gamma ' u ^{ \xpy },\\ \label {eq-yz-xp} \beta ' u ^{ \yw } + \sigma ' u ^{ \wz }& = \tau ' u ^{ \yz }, \text { and }\\ \label {eq-xpz-y} \gamma ' u ^{ \xpy } + \tau ' u ^{ \yz } &= \v ^{ \xpz } . \end {align} \end {linenomath*} \addtocounter {linenumber}{-1} As in the arguments involving $ \gamma $ and $ \tau $, the induction hypothesis yields $ \gamma ' u ^{ \xpy } = \v ^{ \xpy } $ and $ \tau ' u ^{ \yz } = \v ^{ \yz }$. We conclude that $ \tau = \tau ' $. Substituting for $\tau '$ in \cref {eq-yz-xp} and appealing to linear independence of $\{u^{\yw }, u^{\wz }\}$ then yields the desired equalities $\beta = \beta ' $ and $\sigma = \sigma ' $. \par As a consequence of the above argument, for every $y , z \in Y$, take $ \hat {v}^{\yw } $ and $ \hat {v}^{\wz }$ to be the unique vectors in $\R ^{\mbbt }$ that solve the equation $\hat {v} ^{ \yw } + \hat {v} ^{ \wz } = \v ^{ \yz } $. For every $y , z \in Y$, let $\hat {v}^{\yz } = \v ^{\yz }$ and $\hat {v}^{(w,w)} = 0$. Then the matrix $\hat {v}^{\dd } $ with row vectors $\left \{ \hat {v}^{\xy }: x , y \in W \right \}$ is a Jacobi representation of $\hext $. \end {proof} Our proof of \cref {claim-induction-well-defined} shows that the extension to $W$ holds for any initial subsegment of $Y$ consisting of four elements. Our proof thereby accounts for the case where $ X $ is infinite and $w$ is a limit ordinal.\end{proof}
