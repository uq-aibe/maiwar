%==============================================================================
\documentclass[handout,english]{beamer}
%\documentclass{beamer}
%==============================================================================
\usepackage{epigraph}
\usepackage{longtable}
\usepackage{pifont}
\usepackage{media9}  
\usepackage{Preamble_files/beamer_preamble}
\usepackage{comment}
\usepackage{RKsmileys}
\setbeamertemplate{itemize item}{
  \scriptsize\raise1.25pt\hbox{
    \donotcoloroutermaths$\blacktriangleright$
  }
}
\setbeamertemplate{itemize subitem}{
  \tiny\raise1.5pt\hbox{
    \donotcoloroutermaths$\blacktriangleright$
  }
}

\def\checkmark{
  \tikz\fill[scale=0.4](0,.35) -- (.25,0) -- (1,.7) -- (.25,.15) -- cycle;
}
\newcommand{\cmark}{\ding{51}}%
\newcommand{\xmark}{\ding{55}}%

\linespread{1.5}
%\input{../../bib/_addbibresource_Alphabetical_Aligned.tex}

\usetheme{}
\title{Sustainability, Modelling and Regional Transition (in Queensland)}
\author[Patrick O'Callaghan]{Patrick O'Callaghan}
\institute{AIBE, The University of Queensland\\
\today}
\date{}
\AtBeginSection[]
{
\begin{frame}{Table of Contents}
\tableofcontents[currentsection]
\vfill\end{frame}
\normalsize
}
\AtBeginSubsection[]
{
%\begin{frame}{Table of Contents}
%\vskip8pt
%\tableofcontents[currentsection, currentsubsection]
%\vfill\end{frame}
%\normalsize
}


\begin{document}
\maketitle
\section{John Mangan}
\begin{frame}
%==============================================================================
  \frametitle{Why are we here?}
%==============================================================================
  Brief discussion of the history of the project and why we are here.
  \begin{itemize}
    \item why the project is important to AIBE\\
    \item background of regional/sectoral economics at UQ\\
    \item Input-Output modelling\\
    \item the state of CGE modelling in Australia\\
    \begin{itemize}
      \item CoPS, U Victoria (no uncertainty at all or proper dynamics)\\
      \item Warwick McKibben (steady-state, no uncertainty either)\\
    \end{itemize}
  \end{itemize}
\end{frame}
\section{Patrick O'Callaghan}
\begin{frame}
%==============================================================================
  \frametitle{How we arrived at Maiwar: methodology}
%==============================================================================
  Recent literature on macroeconomic production networks: Baqaee--Farhi, \dots
  \begin{itemize}
    \item embeds network modelling in macro models (IO renaissance)
    \item main finding: nonlinear (beyond Cobb-Douglas) effects matter
  \end{itemize}
  Atalay has some good econometric estimates of elasticities
  \begin{itemize}
    \item elasticity of substitution for flows between sectors is about 0.1
  \end{itemize}
  Conclusion: nonlinear effects matter
  \begin{itemize}
    \item {\color{patrickcolor3}
      does not bode well for multi-sectoral steady-state approx.}
    \item this includes Baqaee--Farhi, Atalay, CoPS, McKibben
  \end{itemize}
\end{frame}
\begin{frame}
%==============================================================================
  \frametitle{How we arrived at Maiwar: methodology}
%==============================================================================
  How about ``global'' approximation e.g. Value Function Iteration?
  Building on Scheidegger's machine-learning approach, we found
  \begin{itemize}
    \item grid approach is at least 1 order of magnitude more accurate than 
      steady/ergodic-state approximations, even for 1-good models.
  \end{itemize}
  But:
  \begin{itemize}
    \item trouble with grid approach: $10 ^ {12}$ points for 12 dimensions
    \item approximation is poor outside of grid
    \item VFI is also unstable (poorly suited to multi-sectoral flows)
  \end{itemize}
\end{frame}
\begin{frame}
%==============================================================================
  \frametitle{How we arrived at Maiwar \dots the research question}
%==============================================================================
  My old supervisor (Herakles Polemarchakis) once told me that building a model
  without a question is \emph{the kiss of death}.

  Reading group: Aarushi, Marian, Patrick Duenow and I, late 2020.
  \begin{itemize}
    \item Economy-wide implications of mental health: with Patrick D
    \item COVID impact
    \item In 2022, settled on 2050 net-zero carbon emission targets.
    \begin{itemize}
      \item Most economists think Australia will be better off, but
      \item Adams 2021, CoPS: Qld -6\% GSP and -100k jobs \emph{rel. to base}.
      \item What about Qld targets over and above those of Australia?
    \end{itemize}
  \end{itemize}
\end{frame}

\begin{frame}
%==============================================================================
  \frametitle{MAIWAR \small(Modelling Australian Industry With AMPL Regions)}
%==============================================================================
  \begin{enumerate}\footnotesize
    \item Flexible yet Fast: without steady-state approx
    \item Look-forward property: flow of state-action dependent rewards
  \[
    V_{0}(\omega_{0}) = r_{0}(\omega_{0}, a_{0}) + \cdots
    + r_{9}(\omega_{9}, a_{9}) + V_{10}(\omega_{10})
  \]
    \item Uncertainty: easy way to improve on CoPS
    \item Investment/saving behaviour: Euler equations: CGE Dixon--Rimmer 2020
    \item Data: BLADE, calibration, econometrics
    \item Robust/Reliable: works with a variety of set-ups
    \item Accurate/Accessible: John as end-user, as open source as possible
    \item Modern yet Trustworthy: best-in-class knowledge, 2+ solvers
    \item Scalable: at least to 8 regions and 20 sectors
  \end{enumerate}\normalsize
\end{frame}

\begin{frame}
%==============================================================================
  \frametitle{Cai--Judd's SCEQ: \small Simple (yet Powerful) Certainty Equivalent Method}
%==============================================================================
  E.g. Irreversible risk:  one-off, permanent shock to output.

  Loosely resembles tipping points: each year, chance of ice-shelf \dots

  Yields 28 paths to 2050:
    \begin{itemize}
    \item path where shock never happens
    \item path where shock happens in 2023;
    \item path where shock happens in 2024;
    \item \dots
    \end{itemize}
  Agent(s) make 10-year plans at each time $t$ along a path:\\
  
  Balance consumption today vs uncertain consumption tomorrow.

  Once plan $t$ is made, time reveals state $\omega_{t+1}$ $\rightarrow$ new plan.

\end{frame}

\begin{frame}
%==============================================================================
  \frametitle{Our contribution: multi-sectoral flows}
%==============================================================================
  \[
    k_{t + 1, j} = (1 - \delta) k_{t, j} + s_{t, j}
  \]
  where $s_{tj}$ is a CES function of intermediate flows.
  \[
    s_{t, j} = \left(\sum_{i} \sigma_{i, j} S_{t, i, j} ^ {\rho} \right)
      ^ {\frac{1}{\rho}}
  \]
  From Atalay's model: $\rho = \frac{0.1 - 1}{0.1} = -9$.

  Difficulty: 8 Regions, 20 Sectors, 10 LookForward: $8 * 20^2 * 10 = 32,000$ flows.

  Jacobi Equation:
    \[
      S_{ij} / \sigma_{ij} = 
    \]
\end{frame}
\begin{frame}
%==============================================================================
  \frametitle{The solution for each path for Investment, Kapital and Labour}
%==============================================================================
  

\end{frame}

\begin{frame}
%==============================================================================
  \frametitle{Cai--Judd: under the bonnet}
%==============================================================================
  

\end{frame}

\begin{frame}
%==============================================================================
  \frametitle{Why CGE modelling?}
%==============================================================================

  \begin{itemize}
    \item<+-|alert@+>{\color<.>{blue}
      "Old-fashioned", "black-box", "intractible", \dots
    }
    \item<+-|alert@+>{\color<.>{blue}
      Yet industry demands "CGE" modelling and uses it as a basis for key 
      decisions.
    }
    \item<+-|alert@+>{\color<.>{blue}
      A guide to quantifying the broader repurcussions of sector-specific
      shocks.
    }
    \item<+-|alert@+>{\color<.>{blue}
      A guide to analysing the implications of 
    }
  \end{itemize}
\vfill
\end{frame}
\begin{frame}
%==============================================================================
\frametitle{Macroeconomics with networks}
%==============================================================================

\end{frame}

\begin{frame}
%==============================================================================
\frametitle{Why the focus on uncertainty?}
%==============================================================================

There is pretty strong evidence that the rise in uncertainty is a significant factor holding back the pace of recovery now. [...] research shows that heightened uncertainty slows economic growth, raises unemployment, and reduces inflationary pressures. [...] There is no question that slow growth, high unemployment, and significant uncertainty are challenges for monetary policy.
\end{frame}

%\begin{frame}
%\frametitle{{\color{patrickcolor3}
%  Example: 2050 net-zero emissions targets
%}
%}
%\uncover<+-|alert@+>{\color<.>{blue}
%  In 2017, Qld set a { \color{magenta} \href{https://www.vision6.com.au/em/message/email/view.php?id=1398053&a=79404&k=VZGdbxVB3M5LdOkRxoA2ohzW3lZeME6xaZFD4f5Ic-o}{``state target for net-zero emissions by 2050''}
%  }.
%}
%\begin{itemize}
%\item<+-|alert@+>{\color<.>{blue}
%        Why not simply enforce the Australia-wide net-zero target?
%        }
%\end{itemize}
%
%\uncover<+-|alert@+>{\color<.>{blue}
%        Likely answer: green credentials beyond the national policy \dots
%      }
%\begin{itemize}
%\item<+-|alert@+>{\color<.>{blue}
%        Norwegian Sovereign Fund divestment of Qld public debt?
%        }
%\item<+-|alert@+>{\color<.>{blue}
%        Exports to EU, US, \dots?
%        }
%  \item<+-|alert@+>{\color<.>{blue}
%        Local public opinion?
%        }
%  \item<+-|alert@+>{\color<.>{blue}
%        Encourage innovation \eg\ hydrogen?
%        }
%\end{itemize}
%\uncover<+-|alert@+>{\color<.>{patrickcolor3}
%  What is the  cost of this policy? (A normative question.)
%}
%\begin{itemize}
%
%  \item<+-|alert@+>{\color<.>{blue} {\color{magenta}\href{https://theconversation.com/economists-back-carbon-price-say-benefits-of-net-zero-outweigh-costs-169939}{link}} (to Conversation article)
%        }
%
%%  \item<+-|alert@+>{\color<.>{blue}
%%        We want a model that can (help) answer such questions.
%%        }
%\end{itemize}
%\vfill\end{frame}
%\begin{frame}
%\frametitle{What zoo of models can answer such questions?}
%\uncover<+-|alert@+>{\color<.>{blue}
%}
%\begin{itemize}
%\item<+-|alert@+>{\color<.>{blue}
%  The preferred policy is an economy-wide carbon price.
%        }
%\end{itemize}
%\uncover<+-|alert@+>{\color<.>{patrickcolor3}
%        Answer: economy-wide models (built on normative principles)
%      }
%      \begin{itemize}
%\item<+-|alert@+>{\color<.>{blue}
%        Leontief Input-Output models (1950s+)
%        }
%\item<+-|alert@+>{\color<.>{blue}
%        Centre of Policy Studies (CoPS) CGE model (1970s+)
%              }
%      \end{itemize}
%\uncover<+-|alert@+>{\color<.>{patrickcolor3}
%        Inter-sectoral models with inter-temporal choice and uncertainty:
%      }
%      \begin{itemize}
%        \item<+-|alert@+>{\color<.>{blue}
%          Original model: Long and Plosser (1983)
%        }
%        \item<+-|alert@+>{\color<.>{blue}
%        More recently: Atalay (2017); Baqaee--Farhi \dots
%        }
%        \item<+-|alert@+>{\color{red}
%        Treasury Intersectoral Model (2019)
%        }
%      \end{itemize}
%      \vfill\end{frame}
%\section{Steady-state models}
%\subsection{CoPS}
%
\begin{frame}
{\color{patrickcolor3} Why not use a CoPS CGE model?}
\begin{itemize}
\item<+-|alert@+>{\color<.>{blue}
Cost of software and of data for the model.
}
\vskip5pt
\item<+-|alert@+>{\color<.>{blue}
  CoPS already have a recent paper on 2050 targets
        }
\end{itemize}
\vskip5pt
\uncover<+-|alert@+>{\color<.>{blue}
Moreover:
\begin{itemize}
\item<+-|alert@+>{\color<.>{blue}
        no proper savings/investment: intertemporal behaviour
        }
        \begin{itemize}
\item<+-|alert@+>{\color<.>{blue}
        leads to strange ``macro-closure conditions''
        }
        \end{itemize}
\item<+-|alert@+>{\color<.>{blue}
        no model of risk/uncertainty and associated behaviour
        }
\end{itemize}
}

\uncover<+-|alert@+>{\color{patrickcolor1}
        CoPS assume current economy is in \emph{Deterministic Steady-State}.
      }
      \vskip5pt
\vfill\end{frame}
\begin{frame}
  {Treasury Intersectoral Model (TIM, 2017)
  }

\uncover<+-|alert@+>{\color<.>{blue}
Part of a new generation of Australian models
}
\vskip5pt
\uncover<+-|alert@+>{\color<.>{blue}
J. Miranda-Pinto of UQ had a hand in TIM (and in our choices)
}
\vskip5pt
\uncover<+-|alert@+>{\color<.>{blue}
  TIM has a sister called EMMA (Macro-econometric forecasting)
  }
\begin{itemize}
\item<+-|alert@+>{\color<.>{blue}
        TIM has proper savings
        }
\item<+-|alert@+>{\color<.>{blue}
        114-sector model of Australia
        }
\item<+-|alert@+>{\color<.>{blue}
        but no risk
         }
\item<+-|alert@+>{\color<.>{blue}
        Deterministic steady state \& {\color{red} we can't access}
      }
\end{itemize}
  \end{frame}
\begin{frame}
{Adapting the Atalay model}
  \uncover<+-|alert@+>{\color<.>{blue}
      Atalay assumes economy in (non-determistic) steady state
      }
\begin{itemize}
  \item<+-|alert@+>{\color<.>{blue}
        pretty complete and quite good empirical foundations
        }
  \item<+-|alert@+>{\color<.>{blue}
        {\color{patrickcolor3} we have full access via Matlab \& Stata}
        }
\end{itemize}
\uncover<+-|alert@+>{\color<.>{patrickcolor1}
      Our adaptation of Atalay is the first model in our suite.
    }
\begin{itemize}
  \item<+-|alert@+>{\color<.>{blue}
        less than 1 second to solve a 20-sector model
        }

  \item<+-|alert@+>{\color<.>{blue}
        regionalise using LGA-level income data via Table Builder
        }
  \item<+-|alert@+>{\color<.>{blue}
        capital flows matrix by adapting a US flows table from 1997 \RKconfused\
        }
  \item<+-|alert@+>{\color<.>{blue}
        Social Accounting Matrix using Current and Capital Accounts}
\end{itemize}
  \end{frame}
\section{``Global'' solution methods}
\begin{frame}
  {\color{blue} Quote from Cai and Judd (Feb, 2021)}

  {\color{blue} Macroeconomists [and CoPS] are often interested in obtaining solutions around the non-stochastic steady state.}

  {\color{patrickcolor3} However in reality, the initial state could be far away from the steady state, and a policymaker may be more interested in the solutions for the initial periods in the forward-looking model than the far future states that could be around the steady state.}

  {\color{patrickcolor1} For example, in environmental and climate change economics \dots}
\end{frame}

\subsection{Value functions}


\begin{frame}\frametitle{
    The current value function $V_{0}$
}
\uncover <+-|alert@+>{\color<.>{blue}
  \begin{align*}
    V_{0}(\omega_{0}) = \max_{a_{\cdot}}& \quad \mathbb E
    \left\{\sum_{t = 0}^{27} \beta^{t} r_{t}(\omega_{t}, a_{t}) + \beta^{28} V_{28}(\omega_{28})\right\}
    \\
    \textup{s.t.}& \quad \omega_{t+1} = g_{t}(\omega_{t}, a_{t}, \varepsilon_{t}), \quad t = 0, \dots, 27
    \\
    & \quad f_{t}(\omega_{t}, a_{t}) \geq 0, \quad t = 0, \dots, 27.
  \end{align*}
}
\uncover<+-|alert@+>{\color<.>{blue}
With current state $\omega_{t}$, action  $a_{t}$, expectation $\mathbb E$, reward $r_{t}$, discount factor $0 < \beta < 1$, terminal value function $V_{28}$, transition law $g_{t}$, error $\varepsilon_{t}$ and feasibility constraints $f_{t}$ on actions.}
\\
\begin{itemize}
\item<+-|alert@+>{\color<.>{blue}
        \emph{\color{patrickcolor1} Once approximated}, iterate over $t$ to get \emph{\color{patrickcolor3} optimal policy} $a^{*}_{\cdot}(\omega_{\cdot})$
                }
\end{itemize}

\vfill\end{frame}
%
%\subsection{Estimating and approximating Value functions}
%\begin{frame}
%  {Estimating/approximating $V_0$ and $V_{28}$}
%  \begin{itemize}
%    \item<+-|alert@+>{\color<.>{blue}
%          Estimate parameters:
%          }
%          \begin{itemize}
%              \item<+-|alert@+>{\color<.>{blue}
%                  Entity data: Jobs in Australia, Consumption Surveys, BLADE
%                  }
%              \item<+-|alert@+>{\color<.>{blue}
%                  Link data: Input-Output Tables, ABS (Elazar) supply-chain, illion + BLADE
%                  }
%              \item<+-|alert@+>{\color<.>{blue}
%                  OECD, \etc : productivity and emissions data
%                  }
%          \end{itemize}
%  \end{itemize}
%\uncover<+-|alert@+>{\color<.>{blue}
%  Three approaches to {\color{patrickcolor1}``globally''  \footnotesize(not Atalay)} approximating $V_{\cdot}$
%}
%\begin{itemize}
%        \item<+-|alert@+>{\color<.>{blue}
%        Scheidegger--Bilionis: Gaussian Process model
%        \begin{itemize}
%          \item<+-|alert@+>{\color<.>{blue}
%               working Python model on GitHub
%                }
%          \item<+-|alert@+>{\color<.>{blue}
%               robust (uses Ipopt)
%                }
%          \item<+-|alert@+>{\color<.>{blue}
%                slow, but Nimrod of UQ HPC will solve this
%                }
%          \item<+-|alert@+>{\color<.>{blue}
%                scratch/sandbox model in Julia
%                }
%          \end{itemize}
%        }
%        \end{itemize}
%\end{frame}
%
%\begin{frame}
%  {Two more novel approaches}
%\uncover<+-|alert@+>{\color<.>{blue}
%  Azinovic--Gaegauf--Scheidegger: Deep Equilibrium Networks
%}
%  \begin{itemize}
%    \item<+-|alert@+>{\color<.>{blue}
%          {\color{patrickcolor3} full access to Python code}, we have basic model
%        }
%    \item<+-|alert@+>{\color<.>{blue}
%          Flexible: approximate around ergodic set \& easily use GPUs
%          }
%    \item<+-|alert@+>{\color<.>{blue}
%          Robust: involves approximation by composing linear functions
%          }
%        \end{itemize}
%\uncover<+-|alert@+>{\color<.>{blue}
%  Cai and Judd: Simulated Certainty Equivalent Approximation
%  }
%  \begin{itemize}
%    \item<+-|alert@+>{\color<.>{blue}
%          {\color{patrickcolor3} full access to GAMS code}, translating to Julia, uses Ipopt
%          }
%  \item<+-|alert@+>{\color<.>{blue}
%          {\color{patrickcolor1} Made for 2050-style questions}
%          }
%    \item<+-|alert@+>{\color<.>{blue}
%          Fast: grid-free without relying on SS or ergodic set
%          }
%          \begin{itemize}
%    \item<+-|alert@+>{\color<.>{blue}
%          simulates paths of length 28, Nimrod useful again (for larger models)
%                  }
%          \end{itemize}
%        \end{itemize}
%\end{frame}
%\section{Summary}
%\subsection{Table}
%\begin{frame}
%\frametitle{}
%
%\uncover<+-|alert@+>{\color<.>{blue}
%
%
%\begin{longtable}{| c | c | c | c | c |}
%\hline
%& Atalay & GP & DEQN & SCEQ
%\\\hline
%SS & \cmark & \xmark &  \xmark & \xmark
%\\\hline
%  ``Global'' & \xmark & \cmark & \cmark & \cmark
%\\\hline
%Gridfree & \cmark & \xmark &  \cmark$^{*}$ & \cmark
%\\\hline
%Robust &  \cmark & \cmark &  \cmark & \cmark
%\\\hline
%Fast$^{\dagger}$ & \cmark & \xmark$^{\ddagger}$ & \xmark$^{\ddagger}$ & \cmark
%\\\hline
%Complete &  \cmark$^{\mathsection}$ & \cmark$^{\mathsection}$ & \cmark$^ {\mathsection} $ & \cmark$^{\mathsection}$
%\\\hline
%
%\end{longtable}
%}
%\begin{itemize}
%  \item[ $ * $]<+-|alert@+>{\color<.>{blue}
%        optional
%        }
%  \item[$\dag$]<+-|alert@+>{\color<.>{blue}
%                my math will simplify:  $20 \times 20 = 400 \mapsto 39$ dimensions
%                }
%  \item[$\ddagger$]<+-|alert@+>{\color<.>{blue}
%                Nimrod and GPU respectively
%                }
%\item[ $ \mathsection $]<+-|alert@+>{\color<.>{blue}
%        various stages of development
%        }
%
%\end{itemize}
%\vfill\end{frame}

\begin{frame}
\frametitle{Thanks for listening!}
\uncover<+-|alert@+>{\color<.>{blue}
Many thanks to the absent members of the team:
}

\begin{itemize}
\item<+-|alert@+>{\color<.>{blue}
        Josh Aberdeen
        }
\item<+-|alert@+>{\color<.>{blue}
        Patrick Duenow
        }
\item<+-|alert@+>{\color<.>{blue}
        Cameron Gordon
        }

\end{itemize}

\vfill\end{frame}


%\printbibliography

\end{document}
